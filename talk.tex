
\documentclass[hyperref={pdfpagelabels=false},ngerman]{beamer}

% stop font warning
\let\Tiny=\tiny
\providecommand\thispdfpagelabel[1]{}

\usepackage[english]{babel}
\usepackage{lmodern}
\usepackage[T1]{fontenc}
\usepackage[utf8]{inputenc}
\usepackage{graphicx,import}
\usepackage{feynmp}
\DeclareGraphicsRule{*}{mps}{*}{} 
\DeclareGraphicsExtensions{.pdf}
\usepackage{amsmath,amssymb,amstext,amsfonts} % mathrsfs
\usepackage{array,booktabs,tabularx}
\usepackage{tikz,tikz-uml,pgf-pie}
\usetikzlibrary{shapes,calc,arrows,positioning}
\usepackage{slashed}
\usepackage{fixltx2e} % textsubscript
\usepackage{multirow}
\usepackage{tcolorbox}
\usepackage{pifont}
\usepackage{xspace}
\usepackage{hyperref}
\hypersetup{colorlinks,linkcolor=,urlcolor=blue}
\usepackage{listings}
\lstset{breaklines=true,
  breakatwhitespace=true,
%  numbers=left,
  numberstyle=\tiny,
  stepnumber=1,
  basicstyle=\ttfamily\footnotesize,
  commentstyle=\ttfamily\color{gray},
  postbreak={\mbox{{$\hookrightarrow$}}\space\space},
  breakindent=10pt,
  breakautoindent=false,
  showspaces=false,
  showstringspaces=false,
  frame=single}

\definecolor{darkgreen}{RGB}{0,176,0}

\newcommand{\cmark}{\ding{51}}%
\newcommand{\xmark}{\ding{55}}%
\newcommand{\fmfvcenter}[1]{\;\vcenter{\hbox{\fmfreuse{#1}}}\;}
\newcommand{\eh}[1]{\,\mathsf{#1}}
\newcommand{\ok}{\textcolor{darkgreen}{\cmark}}
\newcommand{\notok}{\textcolor{red}{\xmark}}
\newcommand{\maybe}{\textcolor{gray}{\cmark}}
\newcommand{\meh}{\textcolor{gray}{\textbf{\huge\lower.1em\hbox{-}}}}
\newcommand{\Lagr}{\mathcal{L}}
\newcommand{\mathi}{\mathsf{i}}
\newcommand{\mycite}[1]{\textcolor{darkgray}{\tiny [#1]}}
\newcommand{\bigcite}[1]{\textcolor{darkgray}{[#1]}}
\newcommand{\dimrep}[1]{\mathbf{#1}}
\newcommand{\dimrepadj}[1]{\mathbf{\overline{#1}}}
\newcommand{\ESSM}{E\textsubscript{6}SSM}
\newcommand{\CESSM}{CE\textsubscript{6}SSM}
\DeclareMathOperator{\tildeRe}{\widetilde Re}
\DeclareMathOperator{\sign}{sign}
\DeclareMathOperator{\re}{Re}
\DeclareMathOperator{\im}{Im}
\renewcommand{\emph}{\textbf}
\newcommand{\dd}{\mathsf{d}}
\newcommand{\myurl}[1]{\href{#1}{#1}}
\newcommand{\Superpot}{\mathcal{W}}
\newcommand{\SuperField}[1]{#1}
\newcommand{\ConjSuperField}[1]{\bar{#1}}
\newcommand{\UY}{\ensuremath{U(1)_{Y}}}
\newcommand{\UN}{\ensuremath{U(1)_{N}}}
\newcommand{\Uem}{\ensuremath{U(1)_\text{em}}}
\newcommand{\SUL}{\ensuremath{SU(2)_\text{L}}}
\newcommand{\SUc}{\ensuremath{SU(3)_\text{c}}}
\newcommand{\SOten}{\ensuremath{{SO(10)}}}
\newcommand{\comma}{,}
\newcommand{\DRbar}{\ensuremath{\overline{\text{DR}}}\xspace}
\newcommand{\MSbar}{\ensuremath{\overline{\text{MS}}}\xspace}
\newcommand{\SM}{\ensuremath{\text{SM}}}
\newcommand{\MSSM}{\ensuremath{\text{MSSM}}}
\newcommand{\pole}{\ensuremath{\text{pole}}}
\newcommand{\tree}{\ensuremath{\text{tree}}}
\newcommand{\fsstar}{\textbf{*}}
\newcommand{\MS}{\ensuremath{M_\text{SUSY}}}
\newcommand{\Zv}{\ensuremath{\backslash\mkern-11.0mu{Z_3}}}
\newcommand{\downrightknickarrow}{\mathrel{\scalebox{1.3}{\rotatebox[origin=c]{180}{$\Lsh$}}}}
\newcommand{\threelinebrace}{$\left. \begin{array}{c} \\ \\ \\ \end{array} \right\rbrace$}
\newcommand{\fivelinebrace}{$\left. \begin{array}{c} \\ \\ \\ \\ \\ \end{array} \right\rbrace$}
\newcommand{\twolinebrace}{$\left. \begin{array}{c} \\ \\ \end{array} \right\rbrace$}
\newcommand{\elevenlinebrace}{$\left. \begin{array}{c} \\ \\ \\ \\ \\ \\ \\ \\ \\ \\ \\ \end{array} \right\rbrace$}

\newcommand{\amu}{a_\mu}
\newcommand{\amuexp}{\amu^{\text{exp}}}
\newcommand{\amuSM}{\amu^{\SM}}
\newcommand{\amuQED}{\amu^{\text{QED}}}
\newcommand{\amuweak}{\amu^{\text{weak}}}
\newcommand{\amuhad}{\amu^{\text{had}}}
\newcommand{\amuMSSM}{\amu^{\MSSM}}
\newcommand{\amuMSSMOneL}{\amu^{\MSSM,\text{1L}}}
\newcommand{\amuMSSMTwoL}{\amu^{\text{2L}}}
\newcommand{\amuMSSMTwoLa}{\amu^{\text{2L(a)}}}
\newcommand{\amuMSSMTwoLb}{\amu^{\text{2L(b)}}}
\newcommand{\amuMSSMTwoLBZC}{\amu^{\text{2L},\chi,}}
\newcommand{\amuMSSMTwoLBZf}{\amu^{\text{2L},\tilde{f}}}
\newcommand{\amuMSSMTwoLHf}{\amu^{\text{2L},f}}
\newcommand{\amuMSSMTwoLHV}{\amu^{\text{2L},\text{bos}}}
\newcommand{\amuMSSMTwoLPhotonic}{\amu^{\text{2L},\gamma}}
\newcommand{\amuMSSMTwoLFSF}{\amu^{\text{2L},f/\tilde{f}}}
\newcommand{\amuMSSMTwoLrest}{\amu^{\text{2L,rest}}}

% set look of slides
\usetheme{Madrid}
\useoutertheme{default}
\useinnertheme{circles}
\usecolortheme{default}
\beamertemplatenavigationsymbolsempty % keine Navigationselemente
\setbeamersize{text margin left = 1cm, text margin right = 1cm}

% define footer
\makeatletter
\setbeamertemplate{footline}
{
  \hfill\hbox{\insertframenumber{} / \inserttotalframenumber\hspace*{4pt}}%
  \vskip3pt%
}
\makeatother
\usecolortheme{tud}

\title{The anomalous magnetic moment of the muon and supersymmetry}

\author{Alexander Voigt}

\date{19.05.2017}

%\institute[Aachen]{RWTH Aachen}
\keywords{Aachen,LHC,muon,magnetic,moment,GM2Calc,FlexibleSUSY,supersymmetry}

%%%%%%%%%%%%%%%%%%%%%%%%%%%%%%%%%%%%%%%%%%%%%%%%%%%%%%%%%%%%%%%%%%%%%%%%%%%%%

\begin{document}

%%%%%%%%%%%%%%%%%%%%%%%%%%%%%%%%%%%%%%%%
\begin{frame}[plain]
  \tikz [remember picture,overlay]
  \node at
    ([yshift=1.3cm,xshift=4cm]current page.south)
    {\includegraphics[height=2cm]{images/RWTH_Logo}};
  \titlepage  
\end{frame}

%%%%%%%%%%%%%%%%%%%%%%%%%%%%%%%%%%%%%%%%
\begin{frame}{Contents}
  \tableofcontents
\end{frame}

\section{Definition and SM prediction}

\begin{frame}{Magnetic moment}
  \begin{align*}
    \vec{m} = g \frac{Qe}{2m_\mu} \vec{s}
  \end{align*}
  g \ldots\ gyromagnetic ratio
  \begin{align*}
    \amu = \frac{g-2}{2}
  \end{align*}
  $\amu$ \ldots\ anomalous magnetic moment
\end{frame}

\begin{frame}{Anomalous magnetic moment in QFT}
  In QFT $\amu$ can be extracted from the $\mu$--$\bar\mu$--$A^\rho$
  3-point function
  \begin{align*}
    \bar{u}(p') \Gamma_{\mu\bar{\mu}A^\rho} u(p) = 
    e \bar{u}(p') \left[ \gamma_\rho F_V(q^2) + (p+p')_\rho F_M(q^2) + \ldots \right] u(p)
  \end{align*}
  in the limit $q^2 \rightarrow 0$:
  \begin{align*}
    g &= 2\left[1 - 2m_\mu F_M(0)\right] & &\Rightarrow & \amu &= - 2m_\mu F_M(0)
  \end{align*}
  \emph{Note:} $a_\mu = 0$ at tree-level $\Rightarrow$ $a_\mu^{1L}$ is
  leading order!
\end{frame}

\newsavebox{\feynmanrules}
\sbox{\feynmanrules}{
\begin{fmffile}{Feynman/FeynmanRules} % file name and path
  \fmfset{thin}{.8pt}
  \fmfset{wiggly_len}{5mm}
  \fmfset{dash_len}{2.5mm}
  \fmfset{dot_size}{1thick}
  \fmfset{arrow_len}{2.5mm}
  \fmfset{curly_len}{2.5mm}

\begin{fmfgraph}(60,60)
  \fmfkeep{mmA}
  \fmftop{v1}
  \fmfbottom{v2,v3}
  \fmf{photon}{v1,c}
  \fmf{electron}{v2,c,v3}
  \fmfdot{c}
\end{fmfgraph}
\begin{fmfgraph*}(60,60)
  \fmfkeep{QED}
  \fmftop{v1}
  \fmfbottom{v2,v3}
  \fmf{photon}{v1,c}
  \fmf{electron}{v2,c1,c,c2,v3}
  \fmffreeze
  \fmf{photon,label=$\gamma$}{c1,c2}
  \fmfdot{c,c1,c2}
\end{fmfgraph*}
\begin{fmfgraph*}(60,60)
  \fmfkeep{weakH}
  \fmftop{v1}
  \fmfbottom{v2,v3}
  \fmf{photon}{v1,c}
  \fmf{electron}{v2,c1,c,c2,v3}
  \fmffreeze
  \fmf{neutralscalar,label=$h$}{c1,c2}
  \fmfdot{c,c1,c2}
\end{fmfgraph*}
\begin{fmfgraph*}(60,60)
  \fmfkeep{weakW2}
  \fmftop{v1}
  \fmfbottom{v2,v3}
  \fmf{photon}{v1,c}
  \fmf{boson,label=$W$}{c1,c}
  \fmf{boson,label=$W$,label.side=left}{c,c2}
  \fmf{electron}{v2,c1}
  \fmf{electron}{c2,v3}
  \fmffreeze
  \fmf{electron}{c1,c2}
  \fmfdot{c,c1,c2}
\end{fmfgraph*}
\begin{fmfgraph*}(60,60)
  \fmfkeep{SMhad}
  \fmftop{v1}
  \fmfbottom{v2,v3}
  \fmf{photon}{v1,c}
  \fmf{electron}{v2,c1,c,c2,v3}
  \fmffreeze
  \fmf{photon}{c1,ci,c2}
  \fmfdot{c,c1,c2}
  \fmfblob{0.5em}{ci}
\end{fmfgraph*}
\begin{fmfgraph*}(60,20)
  \fmfkeep{chiralM}
  \fmfleft{v1}
  \fmfright{v2}
  \fmflabel{$\mu_L$}{v1}
  \fmflabel{$\mu_R$}{v2}
  \fmf{plain}{v1,c,v2}
  \fmfdot{c}
\end{fmfgraph*}
\begin{fmfgraph*}(60,20)
  \fmfkeep{chiralH}
  \fmfleft{v1}
  \fmfright{v2}
  \fmflabel{$\tilde{\mu}_L$}{v1}
  \fmflabel{$\tilde{\mu}_R$}{v2}
  \fmf{neutralscalar}{v1,c,v2}
  \fmfdot{c}
\end{fmfgraph*}
\begin{fmfgraph*}(80,60)
  \fmfkeep{SUSY1LC}
  \fmfleft{v1}
  \fmfright{v2}
  \fmflabel{$\mu$}{v1}
  \fmflabel{$\mu$}{v2}
  \fmf{fermion}{v1,c1}
  \fmf{fermion}{c2,v2}
  \fmf{fermion,left,label=$\chi^\pm_i$}{c1,c2}
  \fmffreeze
  \fmf{neutralscalar,label=$\tilde{\nu}_\mu$}{c1,c2}
  \fmfdot{c1,c2}
\end{fmfgraph*}
\begin{fmfgraph*}(80,60)
  \fmfkeep{SUSY1LN}
  \fmfleft{v1}
  \fmfright{v2}
  \fmflabel{$\mu$}{v1}
  \fmflabel{$\mu$}{v2}
  \fmf{fermion}{v1,c1}
  \fmf{fermion}{c2,v2}
  \fmf{plain,left,label=$\chi^0_i$}{c1,c2}
  \fmffreeze
  \fmf{scalar,label=$\tilde{\mu}_k$}{c1,c2}
  \fmfdot{c1,c2}
\end{fmfgraph*}
\begin{fmfgraph*}(80,80)
  \fmfkeep{SUSY2LaBZChi}
  \fmfleft{v1}
  \fmfright{v2}
  \fmftop{v3}
  \fmf{plain,tension=4}{v1,b1}
  \fmf{plain,tension=4}{b2,v2}
  \fmf{plain}{b1,b2}
  \fmffreeze
  \fmf{neutralscalar,label=$h$,label.side=left,tension=3}{b1,c1}
  \fmf{photon,label=$V$,label.side=left,tension=3}{c2,b2}
  \fmf{plain,left}{c1,c2,c1}
  \fmf{phantom,tension=2}{c1,v3}
  \fmf{phantom,tension=2,label=$\chi$,label.side=right}{c2,v3}
  \fmfdot{b1,b2,c1,c2}
\end{fmfgraph*}
\begin{fmfgraph*}(80,80)
  \fmfkeep{SUSY2LaBZsf}
  \fmfleft{v1}
  \fmfright{v2}
  \fmftop{v3}
  \fmf{plain,tension=4}{v1,b1}
  \fmf{plain,tension=4}{b2,v2}
  \fmf{plain}{b1,b2}
  \fmffreeze
  \fmf{neutralscalar,label=$h$,label.side=left,tension=3}{b1,c1}
  \fmf{photon,label=$V$,label.side=left,tension=3}{c2,b2}
  \fmf{scalar,left}{c1,c2,c1}
  \fmf{phantom,tension=2}{c1,v3}
  \fmf{phantom,tension=2,label=$\tilde{f}$,label.side=right}{c2,v3}
  \fmfdot{b1,b2,c1,c2}
\end{fmfgraph*}
\begin{fmfgraph*}(80,80)
  \fmfkeep{SUSY2Laf}
  \fmfleft{v1}
  \fmfright{v2}
  \fmftop{v3}
  \fmf{plain,tension=4}{v1,b1}
  \fmf{plain,tension=4}{b2,v2}
  \fmf{plain}{b1,b2}
  \fmffreeze
  \fmf{neutralscalar,label=$H$,label.side=left,tension=3}{b1,c1}
  \fmf{photon,label=$V$,label.side=left,tension=3}{c2,b2}
  \fmf{plain,left}{c1,c2,c1}
  \fmf{phantom,tension=2}{c1,v3}
  \fmf{phantom,tension=2,label=$f$,label.side=right}{c2,v3}
  \fmfdot{b1,b2,c1,c2}
\end{fmfgraph*}
\begin{fmfgraph*}(80,80)
  \fmfkeep{SUSY2Labos}
  \fmfleft{v1}
  \fmfright{v2}
  \fmftop{v3}
  \fmf{plain,tension=2}{v1,b1}
  \fmf{plain,tension=2}{b3,v2}
  \fmf{plain}{b1,b2,b3}
  \fmffreeze
  \fmf{neutralscalar,label=$H$,label.side=left,tension=2}{b1,c1}
  \fmf{photon,tension=2}{c1,c2}
  \fmf{photon,tension=0.2}{c1,b2}
  \fmf{photon,tension=2,label=$W$,label.side=left}{c2,b3}
  \fmf{photon,tension=4,label=$\gamma$,label.side=right}{c2,v3}
  \fmfdot{b1,b2,b3,c1,c2}
\end{fmfgraph*}
\begin{fmfgraph*}(80,60)
  \fmfkeep{SUSY2Lbphotonic}
  \fmfleft{v1}
  \fmfright{v2}
  \fmf{fermion}{v1,p1,c1}
  \fmf{fermion}{c2,p2,v2}
  \fmf{fermion,left}{c1,c2}
  \fmffreeze
  \fmf{neutralscalar}{c1,c2}
  \fmf{photon,left}{p1,p2}
\end{fmfgraph*}
\begin{fmfgraph*}(80,60)
  \fmfkeep{SUSY2Lbfsf}
  \fmfleft{v1}
  \fmfright{v2}
  \fmftop{v3}
  \fmf{fermion,tension=2}{v1,c1}
  \fmf{fermion,tension=2}{c4,v2}
  \fmf{scalar}{c1,c4}
  \fmffreeze
  \fmf{plain}{c1,c2}
  \fmf{plain}{c3,c4}
  \fmf{fermion,left,tension=0.01}{c2,c3}
  \fmf{scalar,left,tension=0.01}{c3,c2}
  \fmf{phantom}{v3,c2}
  \fmf{phantom}{v3,c3}
\end{fmfgraph*}
\begin{fmfgraph*}(80,60)
  \fmfkeep{SUSY2Lbrest}
  \fmfleft{v1}
  \fmfright{v2}
  \fmf{fermion}{v1,c1}
  \fmf{plain}{c1,c2}
  \fmf{fermion}{c2,c3}
  \fmf{plain}{c3,c4}
  \fmf{fermion}{c4,v2}
  \fmffreeze
  \fmf{scalar,left}{c1,c3}
  \fmf{scalar,right}{c4,c2}
\end{fmfgraph*}
\end{fmffile}
}

\begin{frame}{SM contribution}
  \begin{align*}
    \amuSM &= \amuQED + \amuweak + \amuhad\\
    \amuQED &= 11658471.895(0.008)\cdot 10^{-10} \\
    \amuweak &= 15.4(0.1)\cdot 10^{-10} \\
    \amuhad &= (692.3(4.2) + 0.7(2.6))\cdot 10^{-10} \\
    \Rightarrow \quad \amuSM &= 11659180.3(0.1)(4.2)(2.6)\cdot 10^{-10}\\
  \end{align*}
  \begin{align*}
    \fmfvcenter{QED} \fmfvcenter{weakH}
    \fmfvcenter{weakW2} \fmfvcenter{SMhad}  \cdots
  \end{align*}
\end{frame}

\begin{frame}{Measurement}
  Spin precession in magnetic field $\vec{B}$ relative to momentum
  \begin{align*}
    \vec{\omega} = \frac{e}{m_\mu} \left[\amu\vec{B}
      - \left(\amu - \frac{1}{\gamma^2 - 1}\right)
      \vec{\beta}\times\vec{E} \right]
  \end{align*}
  vanishes for ``magic'' momentum $p=3.094\eh{GeV}$.
  \begin{align*}
    \amuexp &= 11659208.9(6.3)\cdot 10^{-10} \text{\mycite{E821@BNL}}
  \end{align*}
  $\Rightarrow$
  \begin{align*}
    \amuexp - \amuSM =
    \begin{cases}
      (28.7 \pm 8.0)\cdot 10^{-10} &(3.6\sigma) \text{\mycite{1010.4180}} \\
      (26.1 \pm 8.0)\cdot 10^{-10} &(3.2\sigma) \text{\mycite{1105.3149}}
    \end{cases}
  \end{align*}
\end{frame}

\section{Properties of BSM contributions}

\begin{frame}{BSM contributions}
  generic expected scaling:
  \begin{align*}
    a_\mu^\text{BSM} \propto \frac{m_\mu^2}{M_\text{BSM}^2}
  \end{align*}
  \emph{Reason:} $F_M$ corresponds to chirality flip of muon
  \begin{align*}
    \bar{\mu}\mu = \mu_R \mu_L + \bar{\mu}_L \bar{\mu}_R
  \end{align*}
  Terms in $\mathcal{L}$ corresponding to chirality flip $\propto m_\mu$:
  \begin{align*}
    \fmfvcenter{chiralM} &\qquad\propto m_\mu \\
    \fmfvcenter{chiralH} &\qquad\propto m_\mu \tan\beta
  \end{align*}
  \emph{Potential enhancement factors:}
  $\tan\beta$, $\log\left(\frac{M_\text{BSM}}{m_\mu}\right)$
\end{frame}

\section{MSSM contributions}

\begin{frame}{MSSM contribution -- 1-loop}
  \begin{align*}
    \fmfvcenter{SUSY1LC} \qquad + \qquad \fmfvcenter{SUSY1LN}
  \end{align*}
  \begin{align*}
    \amuMSSMOneL \approx 13\cdot 10^{-10} \left(\frac{100\eh{GeV}}{\MS}\right)^2 \tan\beta\, \sign(\mu M_2)
  \end{align*}
  Properties:
  \begin{itemize}
  \item enhanced by $\tan\beta$
  \item suppressed by $1/\MS^2$
  \item can explain deviation for e.g.\ $\MS = 500\eh{GeV}$, $\tan\beta = 50$
  \end{itemize}
\end{frame}

\begin{frame}{MSSM contribution -- 2-loop}
  \begin{align*}
    \amuMSSMTwoL = \amuMSSMTwoLa + \amuMSSMTwoLb
  \end{align*}
  where
  \begin{align*}
    \amuMSSMTwoLa &= \text{2-loop correction to SM 1-loop diagram}\\
    \amuMSSMTwoLb &= \text{2-loop correction to MSSM 1-loop diagram}
  \end{align*}
\end{frame}

\begin{frame}{MSSM contribution -- $\amuMSSMTwoLa$}
  \begin{align*}
    \amuMSSMTwoLa =
    \amuMSSMTwoLBZC + \amuMSSMTwoLBZf +
    \amuMSSMTwoLHf + \amuMSSMTwoLHV
  \end{align*}
  \begin{align*}
    \amuMSSMTwoLBZC + \amuMSSMTwoLBZf &=
    \fmfvcenter{SUSY2LaBZChi} + \fmfvcenter{SUSY2LaBZsf} = O(10\cdot 10^{-10})
    \\[-3em]
    \amuMSSMTwoLHf &= \fmfvcenter{SUSY2Laf} < 10^{-10}
    \\[-3em]
    \amuMSSMTwoLHV &= \fmfvcenter{SUSY2Labos} < 10^{-10}
  \end{align*}
\end{frame}

\begin{frame}{MSSM contribution -- $\amuMSSMTwoLb$}
  \begin{align*}
    \amuMSSMTwoLb =
    \amuMSSMTwoLPhotonic
    + \amuMSSMTwoLFSF
    + \amuMSSMTwoLrest
  \end{align*}
  \begin{align*}
    \amuMSSMTwoLPhotonic &=
    \fmfvcenter{SUSY2Lbphotonic} \propto \log\frac{\MS}{m_\mu} = -(0.07\ldots 0.09) \amuMSSMOneL
    \\[-2em]
    \amuMSSMTwoLFSF &= \fmfvcenter{SUSY2Lbfsf}
    \propto \log\frac{\MS}{m_\mu} < 0.1 \amuMSSMOneL
    \\[-2em]
    \amuMSSMTwoLrest &= \fmfvcenter{SUSY2Lbrest} = O(2\cdot 10^{-10}), \text{incomplete}
  \end{align*}
\end{frame}

\section{Calculations with GM2Calc, SPheno, FlexibleSUSY}

\begin{frame}{GM2Calc -- state of the art MSSM calculation of $\amuMSSM$}
  \begin{align*}
    \amuMSSM =
    \Bigg[
    \textcolor{darkgreen}{\amuMSSMOneL}
    &+ \textcolor{orange}{\amuMSSMTwoLBZC}
    + \textcolor{orange}{\amuMSSMTwoLBZf}
    + \textcolor{red}{\amuMSSMTwoLHf}
    + \textcolor{red}{\amuMSSMTwoLHV} \\
    &+ \textcolor{darkgreen}{\amuMSSMTwoLPhotonic}
    + \textcolor{orange}{\amuMSSMTwoLFSF}
    + \textcolor{red}{\amuMSSMTwoLrest}
    \Bigg]_{\tan\beta\text{--res.}}
  \end{align*}
  %
  \begin{align*}
    &[\amu]_{\tan\beta\text{--res.}} = [\amu]_{y_f \rightarrow \tilde{y}_f}, \qquad
    \tilde{y}_f = \frac{y_f}{1 + \Delta_f} \qquad
    (f = \mu, \tau, b) \\
    &\Delta_f = \tan\beta\text{-enhanced contributions to $f$ self energy}
  \end{align*}
  %
  \textcolor{darkgreen}{fully implemented}\\
  \textcolor{orange}{approximations implemented}\\
  \textcolor{red}{not implemented}
\end{frame}

\begin{frame}{GM2Calc -- state of the art MSSM calculation of $\amuMSSM$}
  \begin{figure}
    \centering
    \includegraphics[width=0.8\textwidth]{plots/tb_resummation}
  \end{figure}
  BM1 [arXiv:0808.1530]
\end{frame}

\begin{frame}{GM2Calc -- state of the art MSSM calculation of $\amuMSSM$}
  Remaining uncertainty (conservative):
  \begin{align*}
    \Delta \amuMSSM &=
    \Delta\big[\textcolor{orange}{\amuMSSMTwoLBZC} + \textcolor{orange}{\amuMSSMTwoLBZf}\big]
    +\Delta\big[\textcolor{red}{\amuMSSMTwoLHf} + \textcolor{red}{\amuMSSMTwoLHV}\big]
    +\Delta\big[\textcolor{red}{\amuMSSMTwoLrest}\big]\\
    &= 0.3 \left( |a_\mu^{(\chi\gamma H)}| + |a_\mu^{(\tilde{f}\gamma H)}| \right)
    + 0.3 \cdot 10^{-10} + 2 \cdot 10^{-10}\\
    &= 0.3 \left( |a_\mu^{(\chi\gamma H)}| + |a_\mu^{(\tilde{f}\gamma H)}| \right)
    + 2.3\cdot 10^{-10}
  \end{align*}
\end{frame}

\begin{frame}{GM2Calc -- state of the art MSSM calculation of $\amuMSSM$}
  \begin{figure}
    \centering
    \includegraphics[width=0.8\textwidth]{plots/tb_resummation_uncertainty}
  \end{figure}
  BM1 [arXiv:0808.1530]
\end{frame}

\begin{frame}{Comparison GM2Calc and SPheno/FlexibleSUSY}
  \begin{itemize}
  \item GM2Calc: OS scheme (1L + 2L)
  \item SARAH/SPheno: \DRbar scheme (1L)
  \item FlexibleSUSY: \DRbar scheme (1L + 2L QED LL)
  \end{itemize}
  %
  \vspace{1em}
  \emph{1-loop} calculations suffer from:
  \begin{itemize}
  \item uncertainty from choice of renormalization scheme for
    important parameters ($\alpha_{\text{em}}$, $m_{\tilde{\mu}_i}$,
    \ldots), because 1L = LO\\
    $\rightarrow$ resolved at 2-loop level
  \end{itemize}
  %
  \vspace{1em}
  \emph{\DRbar} calculations suffer from:
  \begin{itemize}
  \item full renormalization scale uncertainty, because 1L = LO\\
    $\rightarrow$ resolved at 2-loop level
  \item potentially large 2-loop corrections from quadratically
    enhanced smuon self-energy contributions\\
    $\rightarrow$ avoided in the OS scheme
  \end{itemize}
\end{frame}

\begin{frame}{Comparison GM2Calc and SPheno/FlexibleSUSY}
  \begin{figure}
    \centering
    \includegraphics[width=0.49\textwidth]{plots/OS-vs-DR}
    \includegraphics[width=0.49\textwidth]{plots/OS-vs-DR_splitting}
  \end{figure}
  Note: scale variation = \emph{lower bound} of uncertainty
\end{frame}

\begin{frame}{Status of the MSSM}
  GMSB: [1704.00711] \\
  mSUGRA: dead? \\
  general MSSM: ? \\
\end{frame}

\begin{frame}{Summary and conclusions}
  \begin{itemize}
  \item $a_\mu^\text{BSM} \propto C \frac{m_\mu^2}{M_\text{BSM}^2}$
    \begin{itemize}
    \item large for \emph{low} BSM masses
    \item can be enhanced by model-dependent factors $C \propto
      \tan\beta, \log(\frac{M_\text{BSM}}{m_\mu}), \ldots$
    \end{itemize}
  \item prefer 2-loop calculations over 1-loop calculations\\
    (1-loop = LO, with many ambiguities)
  \item prefer on-shell calculations over \DRbar calculations in the MSSM\\
    (to avoid large corrections to the smuon mass)
  \end{itemize}
\end{frame}

%%%%%%%%%%%%%%%%%%%%%%%%%%%%%%%%%%%%%%%%
% backup slides
%%%%%%%%%%%%%%%%%%%%%%%%%%%%%%%%%%%%%%%%

\begin{frame}[noframenumbering]
  \begin{center}
    \Huge Backup
  \end{center}
\end{frame}

\end{document}
